\documentclass[FIPLY_base.tex]{subfiles}

%\author{Gerald Irsiegler}
%\date{13. März 2016}

\begin{document}
\subsection{Social Media}
\subsubsection{Beschreibung}
Mithilfe der Verknüpfung zu Social Media kann der Benutzer seine Trainingsfortschritte einfach mit seinen Freunden teilen. 
\subsubsection{Verknüpfung mit Facebook}
Die Verknüpfung der App mit Facebook erfolgt über die FacebookSDK und den FacebookLoginButton.
\paragraph{FacebookSDK}\ \\
Mithilfe dieser SDK kann man den Login verwalten und danach im Namen des Benutzers, mit der Einverständnis des Benutzers, posten.
Um die Integrität des Logins und des Datenaustausches zu gewähren, hat jede Applikation welche die Facebook SDK implementiert ihre eigene Serien-Nummer.
Diese wird in der strings.xml gespeichert.
\begin{lstlisting}
<string name="facebook_app_id">1541961082763294</string>
\end{lstlisting}
\paragraph{Facebook Login Button}\ \\
Der Facebook Login Button wird von der Facebook SDK zur Verfügung gestellt und kann ganz einfach in das Layout eingefügt werden.
\begin{lstlisting}
    <com.facebook.login.widget.LoginButton
        android:id="@+id/fbLoginButton"
        android:layout_width="wrap_content"
        android:layout_height="wrap_content"
        android:layout_below="@+id/spGender"
        android:layout_centerHorizontal="true" />
\end{lstlisting}

\newpage
\paragraph{Test Account}\ \\
Da die Applikation während der Entwicklung noch nicht veröffentlicht ist, wird ein Facebook-Test-Account benötigt. \ \\
Die Privatsphäre der Entwickler wird somit auch garantiert, da sonst eine nicht getestete Applikation Postings im Namen der Entwickler, auf deren persönlichen Facebook Seiten machen könnte.

\ \\
Um einen Test Account anzulegen muss man zuerst einen normalen Facebook Account anlegen und ihn dann in den Einstellungen zu einem Test-Account umändern.

\end{document}
