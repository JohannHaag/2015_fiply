\documentclass[FIPLY_base.tex]{subfiles}

%\author{Andreas Denkmayr}
%\date{17. März 2016}

\begin{document}
\subsection{Bundles}
Ein Bundle ist ein Objekt, dass einer Vielzahl von Datentypen einen String zuordnen kann, um Daten zwischen Activities und Fragments zu übertragen.
Die Übergabe des Bundles, und somit der Austausch der Daten, erfolgt über Intents oder Fragment Transactions.
[\citeauthor{soBundles} \cite{soBundles}]

\begin{lstlisting}[caption={Erstellen und Befüllen von Bundles},label=DescriptiveLabel]
Bundle innerBundle = new Bundle();
Bundle outerBundle = new Bundle();

ArrayList<String> arrList = new ArrayList<>();
outerBundle.putStringArrayList("bundleArrayList", arrList);
outerBundle.putInt("bundleInt", 123);
outerBundle.putString("bundleString","Hallo Welt!");
outerBundle.putDouble("bundleDouble", 12.3);
outerBundle.putBundle("anotherBundle",innerBundle);
outerBundle.putBoolean("bundleBoolean",true);
\end{lstlisting}
In diesem Beispiel werden Parameter unterschiedlicher Datentypen zu einem Bundle hinzugefügt.
Das Hinzufügen erfolgt über die zahlreichen put-Methoden. 
Der erste Parameter ist die Kennzeichnung.
Über diese Kennzeichnung kann später wieder auf die Daten zugegriffen werden.
Die zu übertragenden Daten werden als zweiter Parameter in eine put-Methode übergeben.


\subsubsection{Datenübertragung zwischen Fragments}
\begin{lstlisting}[caption={Das Übergeben eines Bundles bei einer FragmentTransaction},label=DescriptiveLabel]
Bundle args = new Bundle();
Fragment fragment = new FTrainingssession();
fragment.setArguments(args);

FragmentManager fragmentManager = getFragmentManager();
FragmentTransaction fragmentTransaction = fragmentManager
	.beginTransaction();
fragmentTransaction.replace(R.id.fraPlace, fragment);
fragmentTransaction.commit();
\end{lstlisting} 
Dieses Bundle wird an ein Fragment angehängt bevor es durch eine FragmentTransaction angezeigt wird.

\newpage
\ \\
\textbf{Zugriff auf die Daten}
\begin{lstlisting}[caption={Der Zugriff auf ein Bundle im neuen Fragment},label=DescriptiveLabel]
Bundle bundleFromBefore = this.getArguments();
Boolean ergBool = bundleFromBefore.getBoolean("bundleBoolean");
String ergString = getArguments().getString("bundleString");
\end{lstlisting}
Die Methode getArguments() ermöglicht einen Zugriff auf ein zuvor übergebenes Bundle.
Für jede put-Methode eines Bundles, existiert auch eine get-Methode, die die Daten des jeweiligen Datentyps zurückliefert.

\subsubsection{Datenübertragung zwischen Activities}
Für die Datenübertragung zwischen Activities gibt es drei mögliche Vorgehensweisen.

\ \\
\textbf{Bundle des Intents}
\begin{lstlisting}[caption={Übertragung durch das Bundle des Intents},label=DescriptiveLabel]
Intent intent = new Intent(this,MainActivity.class);
Bundle intentBundle = intent.getExtras();
intentBundle.putString("bundleString", "Hallo Welt!");
\end{lstlisting} 
Das Bundle des Intents wird durch put-Methoden erweitert.

\ \\
\textbf{Neues Bundle}
\begin{lstlisting}[caption={Übertragung durch einen Intent mit einem neuen Bundle},label=DescriptiveLabel]
Intent intent = new Intent(this,MainActivity.class);
Bundle newBundle = new Bundle();
newBundle.putString("bundleString", "Hallo Welt!");
intent.putExtras(newBundle);
\end{lstlisting}
Es wird ein neues Bundle erstellt und dem Intent mithilfe der putExtras()-Methode hinzugefügt.

\newpage
\ \\
\textbf{Abkürzung durch putExtra()}
\begin{lstlisting}[caption={Abkürzung durch die putExtra()-Methode eines Intents},label=DescriptiveLabel]
Intent intent = new Intent(this,MainActivity.class);
intent.putExtra("bundleString", "Hallo Welt!");
\end{lstlisting} 
Alternativ kann bei Intents auch die putExtra()-Methode verwendet werden. Diese ist für alle Datentypen, die auch ein Bundle unterstützt, verfügbar.
\ \\
\textbf{Zugriff auf die Daten}
\begin{lstlisting}[caption={Zugriff auf die Daten in der neuen Activity durch die getExtras()-Methode},label=DescriptiveLabel]
Bundle bundleFromBefore = this.getIntent().getExtras();
bundleFromBefore.getBoolean("bundleBoolean");
getIntent().getExtras().getString("bundleString");
\end{lstlisting}
Die Methode getIntent().getExtras() ermöglicht einen Zugriff auf ein, durch einen Intent übergebenes, Bundle.

\ \\
Alternativ können die Daten durch getExtra-Methoden abgefragt werden.
\begin{lstlisting}[caption={Zugriff auf die Daten in der neuen Activity durch getExtra()-Methoden},label=DescriptiveLabel]
getIntent().getBooleanExtra("bundleBoolean", true);
getIntent().getStringExtra("bundleString");
\end{lstlisting}
Bei vielen dieser getExtra-Methoden wird als zweiter Parameter ein Standardwert verlangt.


\end{document}
