\documentclass[FIPLY_base.tex]{subfiles}

\author{Gerald Irsiegler}
\date{25. Februar 2016}

\begin{document}
\subsection{Free/Paid Versions}


\subsubsection{Beschreibung}
Es ist eine Basis (Free) Version gratis erhältlich, diese enthält nicht den gesamten Funktionsumhang.
Um alle Funktionen freizuschalten ist eine Gebühr zu zahlen.

\subsubsection{Vorteile}
\begin{itemize}
\item Der Kunde kann sich bevor er Geld investiert einen ersten Eindruck verschaffen und muss die App nicht "blind" kaufen.
\item Da es keine Einstiegs-Barriere gibt, erreicht die App mehr User und wird sich somit schneller verbreiten.
\end{itemize}

\subsubsection{Nachteile}
\begin{itemize}
\item Der Hauptteil aller Apps die dieses Modell wählen, verlieren im Laufe ihrer Lebenszeit Geld. Nur ein sehr kleiner Teil kann sich durchsetzen.
\item Es gibt im Google-Play-Store etwa vier mal mehr gratis Apps, als bezahlte Apps und somit ist es schwerer sich am Markt durchzusetzen.
\end{itemize}

\subsubsection{Implementierung}

Android Studio bietet eine Möglichkeit eine bezahlte und eine gratis in einem Projekt zu entwickeln.
 Dies macht es trivial dieses Konzept in die Realität umzusetzen.

\end{document}
