\documentclass[FIPLY_base.tex]{subfiles}

%\author{Gerald Irsiegler}
%\date{25. Februar 2016}

\begin{document}
\subsection{Gratis/Bezahlte Versionen}


\subsubsection{Beschreibung}
Es ist eine Basis (Gratis) Version gratis erhältlich, diese enthält nicht den gesamten Funktionsumhang.
Um alle Funktionen freizuschalten muss eine Gebühr bezahlt werden.

\subsubsection{Vorteile}
\begin{itemize}
\item Der Kunde kann sich bevor er Geld investiert einen ersten Eindruck verschaffen und muss die App nicht "blind" kaufen.
\item Da es keine Einstiegs-Barriere gibt, erreicht die App mehr User und wird sich somit schneller verbreiten.
\end{itemize}

\subsubsection{Nachteile}
\begin{itemize}
\item Der Hauptteil aller Applikationen die dieses Modell wählen, sind während ihrer Lebenszeit nicht profitabel d.h. dass die Entwicklungskosten den Profit überwiegen. Nur ein sehr kleiner Teil kann sich am Markt durchsetzen.
\item Es gibt im Google-Play-Store etwa vier mal mehr gratis Apps als bezahlte Apps und somit ist es schwerer sich mit einer App am Markt durchzusetzen.
\end{itemize}

\subsubsection{Implementierung}

Android Studio bietet eine Möglichkeit eine bezahlte und eine gratis Version in einem Projekt zu entwickeln.
Dies macht es trivial dieses Konzept in die Realität umzusetzen.

\end{document}
