\documentclass[FIPLY_base.tex]{subfiles}

%\author{Gerald Irsiegler}
%\date{26. Februar 2016}

\begin{document}
\subsection{Die App verkaufen}


\subsubsection{Beschreibung}
Nachdem die App eine größere Benutzerbasis hat, könnte es Interessenten zum Kauf des gesamten Projektes geben.


\subsubsection{Vorteile}
\begin{itemize}
\item Der Gesamtwert einer gut gepflegten App mit großer Userbasis liegt extrem hoch und diese Methode wäre eine der rentabelsten.
\item Flexibilität bei der Weiterentwicklung, die Entwickler können entscheiden ob wir sie das Produkt weiterentwickeln oder nicht. Bei Nichtbeiteiligung ist die Instandhaltung der Applikation aus ihren Händen und das Projekt abgeschlossen.
\end{itemize}

\subsubsection{Nachteile}
\begin{itemize}
\item Falls entschieden wird die Applikation mit der Firma, welche sie erworben hat, weiter zu arbeiten, ist die gestalterische Freiheit der Entwickler eingeschränkt und es ist schwerer Visionen umzusetzen.
\end{itemize}


\subsubsection{Probleme/Schwierigkeiten}
Es werden einige Problem aufgeworfen beim Verkauf einer App an eine größere Firma. Die erste Hürde ist einen Käufer zu finden.
Ohne eine große Userbasis wird das Interesse an dem Produkt sehr niedrig sein, deshalb wird es eine Weile dauern bis dieses Szenario eine Option ist.
Der Verkauf der App würde über einen Dritten erfolgen, da dem Projektteam die Erfahrung in dieser Hinsicht fehlt.
\end{document}
