\documentclass[FIPLY_base.tex]{subfiles}

%\author{Andreas Denkmayr}
%\date{25. Februar 2016}

\begin{document}
\subsection{Donations}
Donations stellen eine Möglichkeit für Benutzer dar den Entwicklern einer App Geld zu spenden, um Ihren Dank auszudrücken oder die Entwicklung der App zu unterstützen.
Es gibt viele Möglichkeiten um die Implementierung von Donations zu unterstützen.\newline
Zu den populärsten Formen zählen Dienste wie PayPal, Flattr, das Implementieren von In-App-Käufen die keinen Gegenwert liefern oder das Erstellen einer kostenpflichtigen App die keine Funktionen beinhaltet und denselben Namen wie die Gratis App und einen Suffix wie z.B.: Donation trägt.


\subsubsection{Flattr}
\begin{quote}
When you're registered to flattr, you add money to your account and set a monthly budget. During the month you flattr creators by clicking the Flattr-button next to their content. At the end of the month, your monthly budget is divided between all the things you flattered and sent to the creators.
\end{quote}[Flattr \cite{flattr}]

\begin{quote}
Flattr can be used as a complement to accepting donations. Or to having advertising. Or to help getting donations you never get for your open source software, blog, music, film, game etc etc.
\end{quote}[Flattr \cite{flattr}]

\ \\
Flattr eignet sich gut um Spenden durchzuführen, da dieses Vorgehensmodell Entwickler direkt unterstützt anstatt Geld für eine bestimmte Leistung entgegen zu nehmen. 

\newpage
\subsubsection{PayPal}
PayPal und deren Mobile Payment Libraries unterstützen die einfache Implementierung eines ''Pay with Paypal''-Buttons über den Käufe mittels eines PayPal-Account durchgeführt werden können.
Da wir unsere App aber in den Google Play Store stellen wollen stehen wir vor dem Problem, dass die Einbindung von Donations in Apps, die über den Play Store vertrieben werden, nur sehr vage in den Google Play-Programmrichtlinien für Entwickler beschrieben sind.

\begin{quote}
Käufe im Store: Entwickler, die Gebühren für Apps und Downloads bei Google Play erheben, müssen dies über das Zahlungssystem von Google Play tun.
\end{quote}[Google Play \cite{gpDevContentPolicy}]

\begin{quote}
Here are some examples of products not currently supported by Google Play In-app Billing: [...]
One time-payments, including peer-to-peer payments, online auctions, and donations.
\end{quote}[Google Play \cite{gpInAppBilling}]
\ \\
\subsubsection{Donations über In-App-Käufe}
Nach diesem Modell werden In-App-Käufe zur Verfügung gestellt, die dem Benutzer die Möglichkeit geben Geld zu bezahlen, ohne einen Gegenwert zu erhalten. 
Die In-App-Käufe werden im Kapitel Freemium näher beschrieben.
\ \\
\subsubsection{Zweite kostenpflichtige App}
Es besteht die Möglichkeit eine zweite App zu erstellen die selbst keine Funktionen beinhaltet und denselben Namen wie die Gratis App + einen Suffix wie z.B.: (Donation) trägt.
In diesem Falle kann ein zufriedener Benutzer diese zweite App kaufen und so den Entwickler der Gratis App unterstützen.

\end{document}